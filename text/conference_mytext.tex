\documentclass[conference]{IEEEtran}
\IEEEoverridecommandlockouts
% The preceding line is only needed to identify funding in the first footnote. If that is unneeded, please comment it out.
\usepackage{cite}
\usepackage{amsmath,amssymb,amsfonts}
\usepackage{algorithmic}
\usepackage{graphicx}
\usepackage{textcomp}
\usepackage{xcolor}
\def\BibTeX{{\rm B\kern-.05em{\sc i\kern-.025em b}\kern-.08em
    T\kern-.1667em\lower.7ex\hbox{E}\kern-.125emX}}
\begin{document}

\title{Audio restoration using plug-and-play approach\\
{}
\thanks{Identify applicable funding agency here. If none, delete this.}
}

\author{\IEEEauthorblockN{Michal Švento}
\IEEEauthorblockA{\textit{dept. name of organization (of Aff.)} \\
\textit{ Brno University of Technology}\\
Brno, Czech Republic \\
212584@vut.cz}
\and
\IEEEauthorblockN{Ondřej Mokrý}
\IEEEauthorblockA{\textit{Signal Processing Laboratory} \\
\textit{ Brno University of Technology}\\
Brno, Czech Republic \\
xmokry12@vut.cz}

}

\maketitle

\begin{abstract}
This document is a model and instructions for \LaTeX.
This and the IEEEtran.cls file define the components of your paper [title, text, heads, etc.]. *CRITICAL: Do Not Use Symbols, Special Characters, Footnotes, 
or Math in Paper Title or Abstract.
\end{abstract}

\begin{IEEEkeywords}
speech enhancement, deep learning, Douglas-Rachford algorithm
\end{IEEEkeywords}

\section{Introduction}

Audio enhancement tasks mostly face problems like missing or damaged samples, noise, or clipping.
If we consider speech signal, we should not avoid the intelligibility problems.
Each problem has developed its own way of enhancing the signal.
Nowadays, the bestway to differentiate algorithms is with two categories:
conventional (autoregressive models, sparsity-based) and solutions using deep learning.

In conventional methods dominates Janssen \cite{Janssen1986} and Etter~\cite{Etter1996}.
These approaches are based on autoregressive signal modeling \cite{Mokry2020}.
Sparse signal representation has changed efficiency of restoration, mainly because increase of computing power.
The information hidden in frequency representation (using proper time-frequency analysis) is sparse, i.e. we do not need each spectral coefficient to repair the signal with improved subjective results.
The most advanced works using sparsity are \cite{Adler2012,Kitic2015,Zaviska2019, Mokry2019}.


Deep learning algorithms have also made their own progress in this area.
The most efficient neural network models are autoencoders,
recurrent neural networks (RNNs) and
Generative Adversial Network (GAN).
Current state-of-the-art deep learned algorithms are Speech Enhancement GAN (SEGAN) \cite{Pascual2017}, NSNet \cite{Xia2020}, FullSubNet \cite{Hao2021}. 

In \cite{Chan2016} was introduced Plug-and-Play method for image restoration.
The idea of a hybrid model,
combining conventional approach (convex minimization) with deep learning,
has shown succesful.
Our motivation is to transform this model to audio problems with minor differences.
We replace Alternating Direction Multiplier Method (ADMM) with Douglas-Rachford algorithm (DR~algorithm).
Denoiser will be chosen from state-of-the-art audio denoisers. 
 

\section{Prerequsities}

In this section we introduce our task in mathematical view and compose minimazation task.

\subsection{Task formulation}

We consider column vector $ s \in \mathbb{R}^{L} $ as our observed single-channel signal of length $ L $.



\subsection{Douglas-Rachford algorithm}



\section{Plug-and-Play inpainting}

\subsection{general algorithm}

\subsection{choice of denoiser}

\subsection{Denoisers}

\section{Testing data and evaluation}


\section{Conclusion}

\section*{Acknowledgment}

The preferred spelling of the word ``acknowledgment'' in America is without 
an ``e'' after the ``g''. Avoid the stilted expression ``one of us (R. B. 
G.) thanks $\ldots$''. Instead, try ``R. B. G. thanks$\ldots$''. Put sponsor 
acknowledgments in the unnumbered footnote on the first page.

\bibliographystyle{IEEEtran}
\bibliography{bib_eeict2023}

%\begin{thebibliography}{00}
%\bibitem{b1} G. Eason, B. Noble, and I. N. Sneddon, ``On certain integrals of Lipschitz-Hankel type involving products of Bessel functions,'' Phil. Trans. Roy. Soc. London, vol. A247, pp. 529--551, April 1955.
%\bibitem{b2} J. Clerk Maxwell, A Treatise on Electricity and Magnetism, 3rd ed., vol. 2. Oxford: Clarendon, 1892, pp.68--73.
%\bibitem{b3} I. S. Jacobs and C. P. Bean, ``Fine particles, thin films and exchange anisotropy,'' in Magnetism, vol. III, G. T. Rado and H. Suhl, Eds. New York: Academic, 1963, pp. 271--350.
%\bibitem{b4} K. Elissa, ``Title of paper if known,'' unpublished.
%\bibitem{b5} R. Nicole, ``Title of paper with only first word capitalized,'' J. Name Stand. Abbrev., in press.
%\bibitem{b6} Y. Yorozu, M. Hirano, K. Oka, and Y. Tagawa, ``Electron spectroscopy studies on magneto-optical media and plastic substrate interface,'' IEEE Transl. J. Magn. Japan, vol. 2, pp. 740--741, August 1987 [Digests 9th Annual Conf. Magnetics Japan, p. 301, 1982].
%\bibitem{b7} M. Young, The Technical Writer's Handbook. Mill Valley, CA: University Science, 1989.
%\end{thebibliography}
\vspace{12pt}
\color{red}
IEEE conference templates contain guidance text for composing and formatting conference papers. Please ensure that all template text is removed from your conference paper prior to submission to the conference. Failure to remove the template text from your paper may result in your paper not being published.

\end{document}
